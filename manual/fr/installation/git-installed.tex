\documentclass[manual-fr.tex]{subfiles}
\begin{document}
Il faut alors aller dans un terminal et taper la commande suivante :\\

git clone https://github.com/YoannDupont/SEM.git\\

Cela va créer un dossier de SEM dans le répertoire où est tapée la commande. \\

Il s'agit de la branche GIT (dépôt), qui sert à gérer les différentes versions du logiciel. Il ne faut pas modifier le contenu de ce dossier car cela pourrait causer des problèmes si l'on souhaite le mettre à jour.\\

L'intérêt ici est de pouvoir mettre-à-jour simplement le logiciel en tapant la commande "git pull" dans la branche. Cela mettra à jour uniquement les fichiers qui doivent l'être, ce qui est pratique quand (comme ici) le contenu est assez lourd.
\end{document}
