\documentclass[12pt]{article}

\usepackage[utf8]{inputenc}
\usepackage[T1]{fontenc}
\usepackage[french]{babel}
\usepackage[backend=biber]{biblatex}
\addbibresource{biblio.bib}

\usepackage[usenames,dvipsnames]{color}

\usepackage[colorlinks=true, citecolor=blue, linkcolor=ForestGreen]{hyperref}

\usepackage{macros}
\usepackage[many]{tcolorbox}

\usepackage[top=1in, bottom=1.5in, left=1in, right=1in]{geometry}

\usepackage{listings}
\lstdefinestyle{pieceofcode}{ 
  basicstyle=\bfseries\color{white},
  backgroundcolor=\color{black},
  stepnumber=2,
  identifierstyle=\color{white},
  stringstyle=\color{white},
  keywordstyle=\color{white},
  commentstyle=\color{white},
}

\usepackage{subfiles}

\title{\SEMFull}
\date{}

\begin{document}
    \maketitle
    
    \tableofcontents
    \newpage
    
    \section{Préface}
    \label{sec:preface}
    
        \subsection{Présentation de \SEM}
        \label{subsec:presentation}
        \subfile{fr/preface/presentation}
    
    \section{Installation}
    \label{sec:installation}
    \subfile{fr/installation}

        \subsection{Si GIT est installé}
        \label{ssubec:git-installed}
        \subfile{fr/installation/git-installed}

        \subsection{Si GIT n'est pas installé}
        \label{subsec:git-not-installed}
        \subfile{fr/installation/git-not-installed}

        \subsection{\Wapiti}
        \label{subsec:wapiti}
        \subfile{fr/installation/wapiti}
    
    \section{Corpus, annotations et ressources linguistiques}
    \label{sec:corpus-annotations-resources}

        \subsection{\FTBFull}
        \label{subsec:ftb}
        \subfile{fr/corpus-annotations-resources/ftb}

        \subsection{Jeu d'annotation PoS}
        \label{subsec:tags-pos}
        \subfile{fr/corpus-annotations-resources/tags-pos}

        \subsection{Annotation en chunks}
        \label{subsec:tags-chunking}
        \subfile{fr/corpus-annotations-resources/tags-chunking}

        \subsection{Annotation en entités nommées}
        \label{subsec:tags-ner}
        \subfile{fr/corpus-annotations-resources/tags-ner}

        \subsection{\LeFFFFull}
        \label{subsec:lefff}
        \subfile{fr/corpus-annotations-resources/lefff}
    
    \section{Formats des fichiers}
    \label{sec:file-formats}
    \subfile{fr/file-formats}

        \subsection{fichiers linéaires}
        \label{subsec:file-linear}
        \subfile{fr/file-formats/file-linear}

            \subsubsection{Exemples}
            \label{subsubsec:file-linear-examples}
            \subfile{fr/file-formats/file-linear/examples}

        \subsection{fichiers vectorisés}
        \label{subsec:file-vectorised}
        \subfile{fr/file-formats/file-vectorised}

            \subsubsection{Exemples}
            \label{subsubsec:file-vectorised-examples}
            \subfile{fr/file-formats/file-vectorised/examples}

        \subsection{fichiers SEM}
        \label{subsec:file-sem}
        \subfile{fr/file-formats/file-sem}

            \subsubsection{Exemples}
            \label{subsubsec:file-sem-examples}
            \subfile{fr/file-formats/file-sem/examples}
    
    \section{Utilisation}
    \label{sec:usage}
    \subfile{fr/usage}

        \subsection{annotate}
        \label{subsec:module-annotate}
        \subfile{fr/usage/module-annotate}

        \subsection{chunking\_fscore}
        \label{subsec:module-chunking-fscore}
        \subfile{fr/usage/module-chunking-fscore}

        \subsection{clean}
        \label{subsec:module-clean}
        \subfile{fr/usage/module-clean}

        \subsection{enrich}
        \label{subsec:module-enrich}
        \subfile{fr/usage/module-enrich}

        \subsection{export}
        \label{subsec:module-export}
        \subfile{fr/usage/module-export}

        \subsection{label\_consistency}
        \label{subsec:module-label-consistency}
        \subfile{fr/usage/module-label-consistency}

        \subsection{tagging}
        \label{subsec:module-tagging}
        \subfile{fr/usage/module-tagging}

        \subsection{segmentation}
        \label{subsec:module-segmentation}
        \subfile{fr/usage/module-segmentation}

        \subsection{compile}
        \label{subsec:module-compile}
        \subfile{fr/usage/module-compile}

        \subsection{decompile}
        \label{subsec:module-decompile}
        \subfile{fr/usage/module-decompile}

        \subsection{tagger}
        \label{subsec:module-tagger}
        \subfile{fr/usage/module-tagger}

        \subsection{gui}
        \label{subsec:module-gui}
        \subfile{fr/usage/module-gui}

        \subsection{annotation\_gui}
        \label{subsec:module-annotation-gui}
        \subfile{fr/usage/module-annotation_gui}
    
    \section{Fichiers de configuration}
    \label{sec:config-files}

        \subsection{Pour le module enrich}
        \label{subsec:config-enrich}
        \subfile{fr/config-files/enrich}

            \subsubsection{Les features \textit{arity}}
            \label{subsubsec:feature-arity}
            \subfile{fr/config-files/enrich/feature-arity}

            \subsubsection{Les features \textit{boolean}}
            \label{subsubsec:feature-boolean}
            \subfile{fr/config-files/enrich/feature-boolean}

            \subsubsection{Les features \textit{dictionary}}
            \label{subsubsec:feature-dictionary}
            \subfile{fr/config-files/enrich/feature-dictionary}

            \subsubsection{Les features \textit{directory}}
            \label{subsubsec:feature-directory}
            \subfile{fr/config-files/enrich/feature-directory}

            \subsubsection{Les features \textit{list}}
            \label{subsubsec:feature-list}
            \subfile{fr/config-files/enrich/feature-list}

            \subsubsection{Les features \textit{matcher}}
            \label{subsubsec:feature-matcher}
            \subfile{fr/config-files/enrich/feature-matcher}

            \subsubsection{Les features \textit{rule}}
            \label{subsubsec:feature-rule}
            \subfile{fr/config-files/enrich/feature-rule}

            \subsubsection{Les features \textit{string}}
            \label{subsubsec:feature-string}
            \subfile{fr/config-files/enrich/feature-string}

            \subsubsection{Les features \textit{trigger}}
            \label{subsubsec:feature-triggered}
            \subfile{fr/config-files/enrich/feature-triggered}

        \subsection{Pour le module tagger}
        \label{sec:config-tagger}
        \subfile{fr/config-files/tagger}
    
    \section{Réentraîner \SEM}
    \label{sec:retrain-sem}
    \subfile{fr/retrain-sem}
    
        \subsection{Réentraîner SEM depuis des fichiers déjà annotés}
        \label{subsec:retrain-sem-annotated}
        
            \subsubsection{Lancer la GUI de SEM}
            \label{subsec:launch-sem-gui}
            \subfile{fr/retrain-sem/annotated/launch-sem-gui}
            
            \subsubsection{Sélectionner les données et leur prétraitement}
            \label{subsec:data-and-preprocess}
            \subfile{fr/retrain-sem/annotated/data-and-preprocess}
            
            \subsubsection{Lancer l'entraînement}
            \label{subsubsec:train-SEM-from-annotated}
            \subfile{fr/retrain-sem/annotated/train-sem}
        
        \subsection{Réentraîner SEM depuis des fichiers non annotés}
        \label{sec:retrain-sem-unannotated}
        \subfile{fr/retrain-sem/unannotated}
        
            \subsubsection{Lancer la GUI de SEM pour l'annotation manuelle}
            \label{subsubsec:launch-sem-annotation-gui}
            \subfile{fr/retrain-sem/unannotated/launch-sem-annotation-gui}
        
            \subsubsection{Annoter manuellement avec la GUI de SEM}
            \label{subsubsec:manually-annotate-with-sem}
            \subfile{fr/retrain-sem/unannotated/manually-annotate-with-sem}
        
        \subsection{Utiliser le nouveau modèle}
        \label{subsec:use-new-model}
        \subfile{fr/retrain-sem/use-new-model}
    
    %\bibliographystyle{apa}
    %\bibliography{biblio}
    \printbibliography
\end{document}
