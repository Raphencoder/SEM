\documentclass[manual-fr.tex]{subfiles}
\begin{document}

\begin{itemize}
    \item[] \textbf{description}
        \begin{itemize}
            %\item[] Permet de rajouter des informations à un file vectorisé. Les informations rajoutées sont déclarées dans un file de configuration xml. Ces traits sont détaillés dans la section \ref{subsec:config-enrich}.
            \item[] adds descriptors to a vectorized file. Informations to add are declared in an XML configuration file. Features are explained in section \ref{subsec:config-enrich}.
        \end{itemize}
    \item[] \textbf{arguments}
        \begin{itemize}
            \item[] infile: file
                \begin{itemize}
                    \item[] the input file, in vectorized format.
                \end{itemize}
            \item[] infofile: file
                \begin{itemize}
                    \item[] configuration file where features are declared, XML format.
                \end{itemize}
            \item[] outfile: file.
                \begin{itemize}
                    \item[] The output file, in vectorized format.
                \end{itemize}
        \end{itemize}
    \item[] \textbf{options}
        \begin{itemize}
            \item[] --help ou -h: switch
                \begin{itemize}
                    \item[] displays help
                \end{itemize}
            \item[] --input-encoding: string
                \begin{itemize}
                    \item[] The encoding of the input file. Has priority over --encoding (default: --encoding).
                \end{itemize}
            \item[] --output-encoding: string
                \begin{itemize}
                    \item[] The encoding of the output file. Has priority over --encoding (default: --encoding).
                \end{itemize}
            \item[] --encoding: string
                \begin{itemize}
                    \item[] Encoding of both the input and the output files. Does not have priority (default: UTF-8).
                \end{itemize}
            \item[] --log ou -l: string
                \begin{itemize}
                    \item[] the log level: info, warn or critical (default: critical).
                \end{itemize}
            \item[] --log-file: file
                \begin{itemize}
                    \item[] the file where to log (default: terminal).
                \end{itemize}
        \end{itemize}
\end{itemize}

\end{document}
